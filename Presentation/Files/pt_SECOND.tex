\section{Analysis}
	\subsection{Theoretical}
	
	\subsection{Empirical}
	
	\subsection{Benchmark}
		\begin{frame}[allowframebreaks]{Clustering a Benchmark Data Set}
			\textbf{The Iris Dataset}
			\begin{itemize}
				\item[$\bullet$] Reduced due to qubit limitations on modern hardware
				\item[$\bullet$] Pick $N/k$ points from $2\leq k \leq3$ of the data set's classes
			\end{itemize}
		
			\textbf{Experiments Run}
			\begin{itemize}
				\item[$\bullet$] All the 3 clustering algorithms were tested
				\item[$\bullet$] Experiments are run on 50 subsets of the dataset
			\end{itemize}
		
			\textbf{Results}
			\begin{itemize}
				\item[$\bullet$] $k=2$
				\begin{itemize}
					\item[$\circ$] Trivial case, points are linearly separable
					\item[$\circ$] Classical algorithms perform better than quantum
					\item[$\circ$] Evident as the number of binary variables $(Nk)$ increases
				\end{itemize}
				\framebreak
				\item[$\bullet$] $k=3$
				\begin{itemize}
					\item[$\circ$] \textbf{QA} has similar performance to \textbf{Classical Balanced k-means}
					\item[$\circ$] \textbf{QA} outperforms \textbf{Scikit-Learn} implementation
					\item[$\circ$] Performance of the QA degrades as the problem size increases
				\end{itemize}
			\end{itemize}
			\begin{center}
				\includegraphics[scale=0.45]{Iris_ARI_Table.png}
			\end{center}
		\end{frame}